\documentclass[a4paper,11pt]{article}
\usepackage{amssymb}
\usepackage[utf8]{inputenc}
\usepackage[OT4]{fontenc}
\usepackage[MeX]{polski}
\usepackage[polish,english]{babel} 
\usepackage{amsfonts}
\usepackage{graphicx}
\usepackage{epstopdf}
\usepackage{url}
\usepackage{float}
\usepackage[affil-it]{authblk}
\usepackage{multirow}
\usepackage{array}
\usepackage{tocloft}
\usepackage{listings}
\usepackage{subfiles}
\usepackage{amsmath}
\usepackage{siunitx}
\usepackage{filecontents}
\usepackage{mathtools}
\usepackage[skip=2pt,font=small]{caption}

\addtolength{\textwidth}{3cm}
\addtolength{\hoffset}{-1.5cm}
\addtolength{\textheight}{2cm}
\addtolength{\voffset}{-1.5cm}

\usepackage{fancyhdr}
\pagestyle{fancy}

\fancyhead{}
\fancyhead[RO,LE]{\slshape \rightmark}
\fancyfoot{}
\fancyfoot[LE,RO]{\thepage}

\usepackage{tikz}
\usetikzlibrary{calc,fit,shapes,arrows,patterns}
\usetikzlibrary{positioning,calc}

\def\oran{orange!30}
\definecolor{shine}{rgb}{0,0.243137254902,0.43137254902}
\definecolor{ggreen}{rgb}{0,0.292372549002,0.0}
\definecolor{bblue}{cmyk}{0.637, 0.392, 0, 0.6}
\definecolor{rred}{cmyk}{0, 0.819, 0.756, 0.502}

\newcommand{\shine}[1]{\textcolor{shine}{\textit{#1}}}
\newcommand{\dig}[1]{\textcolor{bblue}{#1}}

\setcounter{tocdepth}{4}
\setcounter{secnumdepth}{4}

\linespread{1.15}

\cftsetindents{section}{0.5in}{0.5in}
\cftsetindents{subsection}{0.5in}{0.7in}
\cftsetindents{subsubsection}{0.5in}{0.9in}
\cftsetindents{paragraph}{0.5in}{1.1in}

\author{Maciej Lewicki}
\title{DoseMatic -- the guide to the statistical methods}

\begin{document}
\newcommand{\va}{\vspace{10pt}}
\newcommand{\vb}{\vspace{ 3pt}}

\maketitle
\vspace{1cm}
Based on:\\
\textbf{INTERNATIONAL ATOMIC ENERGY AGENCY, VIENNA, 2001}\\
\textit{"Cytogenetic Analysis for Radiation Dose Assessment -- A Manual"}\\
\url{http://www-pub.iaea.org/MTCD/publications/PDF/TRS405_scr.pdf}\va\\
...But a little more explicit.
\vspace{3cm}

%%% TABLE OF CONTENTS %%%
\thispagestyle{empty}
\small{
\tableofcontents
}
\thispagestyle{plain}
\pagebreak

\part{Acute Exposure Dose Estimation}
\section{Dose estimation uncertainty calculation}
In order to express the uncertainty of the dose assessment the $95\%$ confidence interval (later refered as CI$95\%$) is chosen as a reasonable limit. It defines an interval of $95\%$ probability of enclosing the true dose.\\
The confidence limits are affected by two sources of uncertainty:
\begin{itemize}
\item Poisson nature of of the yields aberrations
\item and uncertainties of the calibration curve parameters (following normal distribution).
\end{itemize}
The paper refers to three different approaches, none of them being the exact one, but each has a region of usability.\\
At first let us derive the basic statistics associated with the calibration curve:
\begin{itemize}
\item The calibration curve equation:
\begin{equation} \label{eq:curve}
Y(D) = c + \alpha D + \beta D^2
\end{equation}
\item The inverse function, extracting the dose D:
\begin{equation} \label{eq:inverse}
D(Y) = \frac{2}{\beta} \left(  -\alpha + \sqrt{\alpha^2 + 4 \beta (Y-c)} \right)
\end{equation}
\item The calibration curve (eq.\ref{eq:curve}) differentials with respect to fitting parameters:
\begin{equation} \label{eq:curvediff}
\frac{\partial Y}{\partial c} = 1,~~~\frac{\partial Y}{\partial}{\alpha} = D,~~~\frac{\partial Y}{\partial \beta} = D^2
\end{equation}
\item The uncertainty of calibration curve:
\begin{equation} \label{eq:ucurve}
\begin{split}
u_{\textrm{fit}}(Y) = \frac{\partial Y}{\partial c} \cdot u(c) + \frac{\partial Y}{\partial \alpha} \cdot u(\alpha) + \frac{\partial Y}{\partial \beta} \cdot u(\beta)\\
u_{\textrm{fit}}(Y) =~~u(c)~~+~~D\cdot u(\alpha)~~+~~D^2 \cdot u(\beta)
\end{split}
\end{equation}
\item The inverse function (eq.\ref{eq:inverse}) differentials with respect to fitting parameters:
\begin{subequations}
\begin{align}
\frac{\partial D}{\partial c} &= \frac{4}{\sqrt{\alpha^2+4\beta(Y-c)}} \label{eq:invdiffc}\\
\frac{\partial D}{\partial \alpha} &= -\frac{2}{\beta} + \frac{2\alpha}{\beta\sqrt{\alpha^2+4\beta(Y-c)}} \label{eq:invdiffa}\\
\frac{\partial D}{\partial \beta} &= \frac{4(Y-c)}{\beta\sqrt{\alpha^2+4\beta(Y-c)}} - \frac{2\left( \sqrt{\alpha^2+4\beta(Y-c)} -a \right)}{\beta^2} \label{eq:invdiffb}
\end{align}
\end{subequations}
\item Variance of arbitrary variable X:
\begin{equation} \label{eq:var}
\textrm{var}(X) = E\left[ (X-\bar X)^2 \right] = \frac{1}{n} \sum^{n}_{i} (X_i-\bar X)^2
\end{equation}
\item Covariance of arbitrary variables X and Y:
\begin{equation} \label{eq:cov}
\textrm{cov}(X) = E\left[ (X-\bar X)(Y-\bar Y) \right] = \frac{1}{n} \sum^{n}_{i} (X_i-\bar{X})(Y_i-\bar{Y})
\end{equation}
\item Standard error of arbitrary variable X:
\begin{equation} \label{eq:stderr}
\sigma(X) = \sqrt{\textrm{var}(X)} =  \sqrt{ \frac{1}{n} \sum^{n}_{i} (X_i-\bar X)^2 }
\end{equation}
\item Confidence intervals of $95\%$ (CI$95\%$) for given $sigma$ (eq.\ref{eq:stderr}):
\begin{subequations}
\begin{align}
\textrm{lower CI95\%(X)} &= X - 1.96\cdot \sigma(X)\\
\textrm{upper CI95\%(X)} &= X + 1.96\cdot \sigma(X)
\end{align}
\end{subequations}
\end{itemize}

\subsection{Method A}
There are three steps of the procedure:
\begin{enumerate}
\item Calculate the dose $D$ from the inverse curve equation (eq.\ref{eq:inverse}) for a measured yield $Y$.

\item Calculate the variance of dose -- $\textrm{var}(D)$ -- from a given equation:
\begin{equation} \label{eq:Dvar}
\begin{split}
\textrm{var}(X)~~~=~~~
\left( \frac{\partial D}{\partial c} \right)^2 \cdot \textrm{var}(c)~~+~~ 
\left( \frac{\partial D}{\partial \alpha} \right)^2 \cdot \textrm{var}(\alpha)~~+~~ 
\left( \frac{\partial D}{\partial \beta} \right)^2 \cdot \textrm{var}(\beta)~~+ \\
+~~\frac{\partial D}{\partial \alpha}\frac{\partial D}{\partial \beta} \cdot \textrm{cov}(\alpha,\beta)~~+~~ 
\frac{\partial D}{\partial c}\frac{\partial D}{\partial \alpha} \cdot \textrm{cov}(c,\alpha)~~+~~ 
\frac{\partial D}{\partial c}\frac{\partial D}{\partial \beta} \cdot \textrm{cov}(c,\beta)
\end{split}
\end{equation}
Where all the derivatives are defined above, the variances of curve parameters are givevn by the fitting routine and the variance of the yield $Y$ is derived on the assumption of a Poisson distribution (pmf -- probability mass function, $k$ -- the number of counts, $n$ -- number of cells, $\lambda$ -- the mean number of counts):
\begin{equation} \label{eq:Y}
Y~~\equiv~~\frac{1}{n} \sum^n_i Y_i~~\equiv~~\lambda 
\end{equation}
\begin{equation} \label{eq:Ydist}
\textrm{pmf}(Y_i) = P(k, \lambda) = \frac{\lambda^k e^{-k}}{k!}
\end{equation}
Thus the variance and $\sigma$ are equal respectively:
\begin{equation} \label{eq:Yvar}
\textrm{var}(Y) = \lambda,~~~~~\sigma(Y)=	\sqrt{\lambda}
\end{equation}

\item Now let us derive the CI$95\%$:
\begin{equation} \label{eq:DCI}
\begin{split}
\textrm{lower CI}95\%(D) = D_L = D - 1.96\cdot \sigma(D) \\
\textrm{upper CI}95\%(D) = D_U = D + 1.96\cdot \sigma(D)
\end{split}
\end{equation}
\end{enumerate}

\subsection{Method B}
\begin{enumerate}
\item Calculate the dose $D$ from the inverse curve equation (eq.\ref{eq:inverse}) for a measured yield $Y$.
\item Estimate $Y$ error using curve's uncertaintiy (eq.\ref{eq:ucurve}):
\begin{equation} \label{eq:err1}
Y = Y \pm u_{\textrm{fit}}(Y)
\end{equation}
\item From the observed yield estimate its Poisson standard error (eq.\ref{eq:Yvar}):
\begin{equation} \label{eq:err2}
\sigma(Y) = \sqrt{\lambda}
\end{equation}
\item Add the errors from points 2 and 3 (eq. \ref{eq:err1} and \ref{eq:err2}):
\begin{equation} \label{eq:err}
u(Y) = \sqrt{ \left[ u_{\textrm{fit}}(Y) \right]^2 + \lambda }
\end{equation}
\begin{equation} \label{eq:YLU}
Y_L = Y - u(Y),~~~~~Y_U = Y + u(Y)
\end{equation}
\item Now, using the inverse function (eq.\ref{eq:inverse}) calculate $D_U$ and $D_L$:
\begin{equation} \label{eq:DLU}
D_L = D(Y_L),~~~~~D_U = D(Y_U)
\end{equation}
\end{enumerate}

\subsection{Method C}
\begin{enumerate}
\item Assuming Poisson distribution, calculate the CI$95\%$ on the observed yield:
\begin{equation} \label{eq:YCI}
\begin{split}
Y_L = Y - 1.96\cdot \sigma(Y) \\
Y_U = Y + 1.96\cdot \sigma(Y)
\end{split}
\end{equation}
\item Now determine the intersection of the confidence limits $Y_L, Y_U$ with the calibration curve's upper limit and lower limit respectively (or with the curve itself for simplicity):
 \begin{equation} \label{eq:inter1}
\begin{split}
Y_L = Y(D_L) + u(Y)~~~\rightarrow~~~D_L\\
Y_U = Y(D_U) - u(Y)~~~\rightarrow~~~D_U\\
\end{split}
\end{equation}
\begin{equation} \label{eq:inter2}
\begin{split}
D_L = D(Y_L - u(Y)) \\
D_U = D(Y_U + u(Y)) \\
\end{split}
\end{equation}
Or the simplified, straightforward version:
\begin{equation} \label{eq:inter3}
\begin{split}
D_L = D(Y_L) \\
D_U = D(Y_U) \\
\end{split}
\end{equation}
\end{enumerate}

\section{Conclusions}
\textbf{Method A:}
\begin{itemize}
\item Obviously most accurate. Even almost exact -- if it were not for the usually wrong assumption on the normal distribution of the aberrations. This is a problem for low number of aberrations, which means low doses.
\item This method is recommended for large number of scored aberrations (or high doses).
\end{itemize}

\noindent \textbf{Method B:}
\begin{itemize}
\item Suffers from similar defect as above.
\item The region of recommended usage is when the uncertainty of measured yields is similar to the uncertainty of the calibration curve.
\end{itemize}

\noindent \textbf{Method C:}
\begin{itemize}
\item Takes into account the Poisson nature of the number of aberrations distribution.
\item In the basic form overestimates the effect of calibration curve uncertainties.
\item In the simplified form the curve uncertainties are not taken into account at all.
\item To be used when the uncertainty of yields dominates over the uncertainty of the calibration curve.
\end{itemize}


\part{Partial Exposure Dose Estimation}
\section{Calculation of output values}
\subsection{Basic Statistics}
\begin{itemize}
	\item Standard deviation of a sample:
	$$s = \sqrt{\frac{\sum_{i=1}^n{\left(x_i-\bar{x}\right)^2}}{n-1}}$$
	\item Standard error -- based on standard deviation of a sample:
	$$SE = \frac{\sigma}{n} \approx \frac{s}{n} $$
	\item Dispertion index:
	$$D = \frac{\sigma^2}{\mu} \approx \frac{s^2}{\bar{x}}$$
	\item U-test:
	$$u = \left(D-1\right)\cdot \sqrt{2\cdot\left(1-\frac{1}{\bar{x}}\right)} $$
\end{itemize}
\subsection{Dolphin Method}
\begin{itemize}
	\item Calculation of yield using Dolphin method:
	$$\frac{Y}{1-e^{-Y}} = \frac{X}{N-n_0} $$
	$$Yf = \frac{X}{N}$$
	where:\\
	$N$ -- number of cells scored,\\
	$X$ -- number of observed dicentrics,\\
	$n_0$ -- cells free of dicentrics,\\
	$f$ -- fraction of irradiated body,\\
	$Y$ -- yield:
	$$
	Y = W\left(\frac{e^{\frac{X}{n_0-N}}~X}{n_0-N}\right) - \frac{X}{n_0-N}
	$$
	where $W(\cdots)$ is the principal solution of a Lambert W-function.
\end{itemize}
\subsection{Qdr Method}
\begin{itemize}
	\item Calculation of yield using Qdr method:
	$$
	\textrm{Qdr} = \frac{X}{N_u} = \frac{Y}{1-e^{-Y_1-Y_2}}
	$$
	where:\\
	$X$ -- number of dicentrics plus rings,\\
	$N_u$ -- number of damaged cells,\\
	$Y_1$ -- yield of dicentrics plus rings,\\
	$Y_2$ -- yield of acentrics,\\
	Qdr -- yield of dicentrics and rings among damaged cells,\\
	$Y$ - yield:
	$$
	Y = \textrm{Qdr} \cdot \left( 1 - e^{-Y_1-Y_2} \right)
	$$
\end{itemize}

\end{document}


